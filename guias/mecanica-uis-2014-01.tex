\documentclass[a4paper,12pt]{article}
\usepackage[spanish]{babel}
\hyphenation{co-rres-pon-dien-te}
%\usepackage[latin1]{inputenc}
\usepackage[utf8]{inputenc}
\usepackage[T1]{fontenc}
\usepackage{graphicx}
\usepackage{amsmath}

\usepackage[pdftex,colorlinks=true, pdfstartview=FitH, linkcolor=blue, citecolor=blue, urlcolor=blue, pdfpagemode=UseOutlines, pdfauthor={H. Asorey}, pdftitle={Mecánica 2014 - Guía 01} pdfkeywords={vectores}]{hyperref}
\usepackage[adobe-utopia]{mathdesign}

\hoffset -1.23cm
\textwidth 16.5cm
\voffset -2.0cm
\textheight 26.0cm

\begin{document}
\begin{center}
  {\small{Universidad Industrial de Santander - Escuela de Física}}\\
  {\bf{Mecánica Teórica (Asorey)}}\\
  \vspace{0.4cm}
  Guía 01: Caos\\ 2014
\end{center}
\renewcommand{\labelenumi}{\arabic{enumi})}
\renewcommand{\labelenumii}{\arabic{enumii})}

\begin{enumerate}
  \item {\bf{Oscilador armónico}}

    Resuelva numéricamente la ecuación del oscilador armónico simple, $\ddot
    \theta + \theta = 0$ para el caso $\theta_0=0.05$. Luego: 
    \begin{enumerate}
      \item Realice un gráfico de $\theta(t)$, $\dot \theta(t)$ y $p(q)$.
        Verifique la existencia de un ciclo límite en el espacio de fases. 
      \item Sabiéndose un sistema conservativo, estudie la estabilidad de la
        solución a partir de la evolución de la energía total del sistema,
        $E=E_k+E_g$, como función del tiempo.
      \item Compárela con la solución analítica del péndulo mediante un gráfico
        de la diferencia entre las dos soluciones como función del tiempo 
    \end{enumerate}

  \item {\bf{Oscilador de grandes amplitudes}}

    Repita el ejercicio anterior pero para el oscilador más general, $\ddot
    \theta + \sin \theta = 0$, para los casos $\theta_0=0.05$ y $\theta=1$.

  \item {\bf{Péndulo}}

    Resuelva en numéricamente la ecuación del péndulo general, que en su forma
    adimensionalizada responde a la ecuación: $\ddot \theta + c \dot \theta +
    \sin \theta = a \sin(\Omega t)$. Dejando fijos los valores de $c=0.5$ y
    $\Omega=2/3$
    \begin{enumerate} 
      \item Estudie las soluciones para 80 ciclos de forzado para los casos
        $a=0$, $a=0.3$, $a=0.9$, $a=1.15$, $a=1.35$ y $a=1.45$.
      \item Habiendo amortiguación y forzado, analice el comportamiento de la
        energía total del sistema como función del tiempo.
      \item Para cada valor de $a$, dibuje el diagrama de fases e identifique
        los casos caóticos por la existencia de un atractor extraño.
      \item Realice el mapa de Poincaré en cada caso, considerando los puntos
        del espacio de fases al concluir cada ciclo del forzado. Suponga que se
        han superado los transitorios a partir del ciclo de forzado número
        quince.
      \item Elija una de las condiciones que conduce a un sistema caótico, y
        haga un mapa de Poincaré para $10^6$ ciclos de forzado, obteniendo un
        punto en el medio de cada ciclo. Estudie las condiciones de
        auto-semejanza del mapa obtenido, confirmando que se trata de un
        atractor extraño.  
    \end{enumerate}

  \item {\bf{Van der Pol}}

    Resuelva en forma numérica la ecuación del oscilador de Van der Pol, $\ddot
    \theta - \mu \left (1-\theta^2 \right ) \dot \theta + \theta = a \sin
    \Omega t$ en los siguientes casos:
    \begin{enumerate}
      \item $\mu=0.09$, $a=0.$, $\Omega = 2/3$
      \item $\mu=0.09$, $a=1.15$, $\Omega = 2/3$
      \item $\mu=8.53$, $a=0.$, $\Omega = 0.628$
      \item $\mu=8.53$, $a=1.15$, $\Omega = 0.628$
    \end{enumerate}
    En todos los casos, realice un gráfico de $\theta(t)$, el diagrama del
    espacio de fases, y estudie el comportamiento de la energía como función
    del tiempo.

  \item {\bf{Edmon-Pullen}}

    Para el Hamiltoniano de Edmon-Pullen, 
    \[\mathcal{H} = \frac{1}{2m} \left ( p_x^2 + p_y^2 \right ) + \frac{m \omega_0^2}{2} \left ( q_x^2 + q_y^2 + \frac{1}{\lambda^2} q_x^2 q_y^2 \right ), \]
    obtenga las ecuaciones de movimiento adimensionalizadas y luego resuelva
    los siguientes puntos para los casos con energía $E=5$, $E=20$ y $E=100$
    (unidades arbitrarias).
    \begin{enumerate}
      \item A partir de las ecuaciones de movimiento adimensionalizadas,
        obtenga un sistema de ecuaciones diferenciales ordinarias de primer
        orden apto para ser resuelto utilizando Runge-Kuta de cuarto orden.
      \item Obtenga un conjunto de condiciones iniciales. Teniendo presente que
        el sistema presenta simetría de intercambio ante los índices
        $x\leftrightarrow y$, evite utilizar condiciones simétricas para las
        condiciones inciales. Utilice la energía del sistema como un vínculo.
      \item Estime un valor razonable para el tiempo de simulación, y resuelva
        numéricamente las ecuaciones de movimiento. Verifique la estabilidad de
        las soluciones analizando la evolución temporal de la energía total y
        compárela con el valor definido para la energía en cada caso. 
      \item Estudie los espacios de fases para cada caso. 
      \item Realice un mapa de Poincaré $(q_y, p_y)$ para la condición $x=0$
        (recuerde que para soluciones numéricas, en lugar de igualdad debe
        considerar un entorno del 0, por ejemplo $|x|<10^{-3}$. Este valor
          dependerá del paso numérico considerado.).
      \item Utilice un algoritmo de Transformada Rápida de Fourier Discreta
          (DFFT), como por ejemplo {\texttt{np.fft}}
          (\href{http://docs.scipy.org/doc/numpy/reference/routines.fft.html}{http://docs.scipy.org/doc/numpy/reference/routines.fft.html}),
          para buscar las frecuencias características del sistema. Analice cada
          uno de los tres casos y a partir de los resultados obtenidos
          establezca una relación entre el número de frecuencias
          características y el comportamiento del sistema.
        \item Utilizando el método de las pendientes, obtenga el valor del
          coeficiente de Lyapunov en cada uno de los casos estudiados (recuerde
          que debe mantener el valor de la energía constante y cambiar
          ligeramente una de las condiciones iniciales).
    \end{enumerate}
  \item {\bf{Mapeos}}

    El mapeo de Hassell está dado por la ecuación 
    \[ y_{i+1} = \frac{k_1 y_i}{\left ( 1 + k_2 y_i \right )^{k_3}}, \]
    donde $y_i$ representa a la población actual, y $k_1$, $k_2$ y $k_3$ son
    los parámetros de la ecuación.
    Para el caso $k_2=1$, estudiaremos la evolución temporal de la población en el espacio de parámetros $k_1$ y $k_3$. Para ello: 

    \begin{enumerate}
      \item Escriba un código para resolver la evolución temporal del mapeo de
        Hassell, considerando $150$ ciclos para superar los efectos
        transitorios dados por las condiciones iniciales.
      \item Obtenga los diagramas de bifurcación para los casos $k_3=1,2,3,4,5
        6$ con $k_1$ variando entre $1$ y $1000$ en pasos de $0.5$ de ancho,
        graficando $50$ valores de $y_i$ con $i=151,\ldots,200$ como función de
        $k_1$, para los seis casos considerados en $k_3$. Estudie para cada uno
        de los seis casos el comportamiento del sistema, la existencia de
        bifurcaciones, la transición a regímenes caóticos y la existencia de
        islas de estabilidad. 
      \item Para cada uno de los seis casos anteriores, obtenga la evolución de
        los coeficientes de Lyapunov como función de $k_1$ utilizando el método
        de la derivada. Suponga un valor de $10000$ para la evolución final del
        sistema. Haga un gráfico del coeficiente de Lyapunov superpuesto sobre
        el diagrama de bifurcación en cada caso. 
    \end{enumerate}
\end{enumerate}
\end{document}
