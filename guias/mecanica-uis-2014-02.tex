\documentclass[a4paper,12pt]{article}
\usepackage[spanish]{babel}
\hyphenation{co-rres-pon-dien-te}
%\usepackage[latin1]{inputenc}
\usepackage[utf8]{inputenc}
\usepackage[T1]{fontenc}
\usepackage{graphicx}
\usepackage{amsmath}

\usepackage[pdftex,colorlinks=true, pdfstartview=FitH, linkcolor=blue, citecolor=blue, urlcolor=blue, pdfpagemode=UseOutlines, pdfauthor={H. Asorey}, pdftitle={Mecánica 2014 - Guía 02}]{hyperref}
\usepackage[adobe-utopia]{mathdesign}

\hoffset -1.23cm
\textwidth 16.5cm
\voffset -2.0cm
\textheight 26.0cm

\begin{document}
\begin{center}
  {\small{Universidad Industrial de Santander - Escuela de Física}}\\
  {\bf{Mecánica Teórica (Asorey)}}\\
  \vspace{0.4cm}
  Guía 02: Fuerzas Centrales\\ 2014
\end{center}
\renewcommand{\labelenumi}{\arabic{enumi})}
\renewcommand{\labelenumii}{\arabic{enumii})}

\begin{enumerate}
  \item {\bf{Choque cometario}}

    Un planeta de masa $M$ posee una órbita de excentricidad $\epsilon = 1
    - \alpha$, con $\alpha \ll 1$, alrededor del Sol. Suponga que el
    movimiento del Sol puede ser obviado y sólo actúan las fuerzas
    gravitacionales. Cuando el planeta se encuentra en el afelio es
    golpeado por un cometa de masa $m$, con $m \ll M$, que viajaba en
    dirección tangencial. Suponiendo que la colisión fue totalmente
    inelástica, obtenga la energía cinética mínima que el cometa debe
    tener para cambiar la órbita del planeta a una parábola.

  \item {\bf{Órbita parabólica y circular}}

    Para órbitas parabólicas y circulares en un potencial atractivo de la
    forma $U=r^{-1}$, ambas con el mismo momento angular $L$, muestre que
    la distancia al perihelio de la parábola es la mitad del radio de la
    órbita circular. Luego, verifique que la velocidad de una partícula en
    un punto cualquiera de la órbita parabólica es $\sqrt{2}$ veces más
    grande que la velocidad en el mismo punto pero para una órbita
    circular.

  \item {\bf{Sección eficáz de la Tierra}}

    Analice y discuta el trabajo de J.B. Tatum, ``The Capture
    Cross-Section of Earth for Errant Asteroids'', JRASC {\bf 91}, 276
    (1997),
    \href{http://adsabs.harvard.edu/abs/1997JRASC..91..276T}{1997JRASC..91..276T}.

  \item {\bf{Planeta X}} \label{X}

    Suponga que el ``Planeta X'' existe y es un gemelo de la Tierra y está
    ubicado en la misma órbita que la Tierra pero de forma tal que cuando
    la Tierra está en el perihelio, el planeta X está en el afelio. Ahora
    bien, dado que la órbita es elíptica, el ocultamiento del planeta X no
    siempre es efectivo al ser visto desde la Tierra. Calcule, a primer
    orden de aproximación en $\epsilon$, la máxima separación angular
    entre nuestro planeta gemelo y el Sol visto desde la Tierra. ¿Podría
    ser visible desde la Tierra?

  \item {\bf{Año anómalo}}

    El período de la Tierra entre dos tránsitos sucesivos por el perihelio
    se denomina el ``año anómalo'' y es de $T_A=365.2596$ días solares
    medios. Sabiendo que la excentricidad de la órbita terrestre es
    $\epsilon_\oplus = 0.0167504$ y suponiendo que el movimiento es
    perfectamente Kepleriano, calcule (numéricamente si fuera necesario)
    la apertura angular de la posición de la Tierra medida desde el
    perihelio cuando el tiempo transcurrido es igual a $T_A/4$.

  \item {\bf{Puntos de Lagrange}}

    El problema de 3 cuerpos restringido consiste en dos masas $M_1$ y
    $M_2$ en órbita circular, una alrededor de la otra, y un tercer cuerpo
    de masa $m$, $m \ll M_1$ y $m \ll M_2$, de manera que el efecto
    gravitatorio de este cuerpo sobre los dos mayores puede ser
    despreciado.

    Los puntos de Lagrange (o puntos de Libración), son aquellos puntos
    del espacio (identificados como $L_1, \ldots, L_5$) donde el potencial
    total presenta un extremo y, por lo tanto, la fuerza neta que
    experimenta el cuerpo de masa $m$ es nula.

    A partir del Lagrangiano del sistema para la masa $m$, $\mathcal{L} =
    \frac12 m \left (\dot{r}^2 + r^2 \dot{\theta}^2 \right ) -
    V(r,\theta,t)$, y moviéndonos a un sistema comóvil con los cuerpos que
    gira con frecuencia angular $\omega$, los dos cuerpos masivos parecen
    estar en reposo. Es posible moverse a este sistema mediante la
    transformación $\theta'=\theta + \omega t$. 
    
    \begin{enumerate} 
      \item En el sistema comóvil y en coordenadas cilíndricas $(\rho,
        \theta=\theta'-\omega t, z)$ con origen en el centro de masas,
        obtenga el nuevo Lagrangiano del sistema
      \item Identifique en el Lagrangiano los términos de Coriolis y
        Centrífugo.
      \item Obtenga las ecuaciones de movimiento a partir de las
        ecuaciones de Lagrange, y encuentre las cinco posiciones donde se
        verifica $\dot{\rho}=\dot{z} = \dot{\theta} = 0$. Estos son los
        cinco puntos de Lagrange.
      \item Perturbe las soluciones anteriores y determine si los puntos
        encontrados son estables ($L_4$ y $L_5$) o inestables ($L_1$,
        $L_2$ y $L_3$).
      \item Encuentre las posiciones de los puntos de Lagrange en el
        sistema Tierra-Sol y Sol-Júpiter.
      \item Investigue sobre la importancia de los puntos de Lagrange para la ciencia aeroespacial
    \end{enumerate}

  \item {\bf{Puntos de Lagrange Numéricos}}

    Utilizando un código basado en el algoritmo de leapfrog, resuelva
    numéricamente la órbita de la Tierra en torno al Sol. Utilice los
    principios de conservación para obtener las condiciones iniciales
    cuando la Tierra se encuentra en el afelio. Luego, incluya un tercer
    cuerpo de masa despreciable frente a las masas de la Tierra y el Sol,
    y verifique las soluciones obtenidas en el punto anterior y la
    estabilidad de las mismas. 

  \item {\bf{Planeta X, el regreso}}

    Utilizando un algoritmo leapfrog, resuelva numéricamente la órbita del
    Sistema Planeta X-Sol-Tierra. Suponga que el Planeta X es idéntico a
    la Tierra en todos los aspectos. Para las posiciones iniciales,
    suponga que a tiempo $t=0$, la Tierra se encuentra en el perihelio y
    el planeta X se encuentra en el afelio, y determine las
    correspondientes velocidades de cada planeta en dichos puntos (sólo
    para este punto desprecie las interacciones gravitatorias entre los
    planetas).

  \item {\bf{Cúmulo globular}}

    Modifique el código de leapfrog para obtener un código que funcione
    con el algoritmo Predictor-Corrector de Hermite con paso temporal
    variable. Luego, estudie la evolución de un cúmulo globular esférico
    con 50 estrellas de masas $m=1$. Como condiciones iniciales,
    suponga que todas las estrellas están inicialmente en reposo
    ({\emph{cold start}}) y se encuentran distribuidas uniformemente en el
    interior de una esfera de radio $R=1$ (código {\texttt{cold-start.py}}).
    En el sistema de unidades donde $G=1$, deje el código evolucionar entre
    $t=0$ y $t=1$, con un paso temporal variable cuya base es $0.05$ (de esta
    forma, el $dt=0.05\times \tau$, con $\tau$ igual al tiempo de paso
    variable).

  \item {\bf{Cúmulo globular y agujero negro}}

    Imagine que en el centro del cúmulo globular anterior se encuentra un
    agujero negro de masa $M=1000$ en reposo. Obtenga la evolución del cúmulo
    con las mismas constantes y parámetros del caso anterior. 

\end{enumerate}
\end{document}
