\documentclass[a4paper,12pt]{article}
\usepackage[spanish]{babel}
\hyphenation{co-rres-pon-dien-te}
%\usepackage[latin1]{inputenc}
\usepackage[utf8]{inputenc}
\usepackage[T1]{fontenc}
\usepackage{graphicx}
\usepackage{amsmath}

\usepackage[pdftex,colorlinks=true, pdfstartview=FitH, linkcolor=blue, citecolor=blue, urlcolor=blue, pdfpagemode=UseOutlines, pdfauthor={H. Asorey}, pdftitle={Mecánica 2014 - Guía 03}]{hyperref}
\usepackage[adobe-utopia]{mathdesign}

\hoffset -1.23cm
\textwidth 16.5cm
\voffset -2.0cm
\textheight 26.0cm

\begin{document}
\begin{center}
  {\small{Universidad Industrial de Santander - Escuela de Física}}\\
  {\bf{Mecánica Teórica (Asorey)}}\\
  \vspace{0.4cm}
  Guía 03: Formulación Hamiltoniana\\ 2014
\end{center}
\renewcommand{\labelenumi}{\arabic{enumi})}
\renewcommand{\labelenumii}{\arabic{enumii})}

\begin{enumerate}
  \item {\bf{Partícula en una caja}}

    Una partícula está confinada en una caja unidimensional. Los bordes de la
    caja se están desplazando hacia el centro con una velocidad mucho menor que
    la velocidad de la partícula en la caja. Resuelva los siguientes puntos
    utilizando primero la formulación Lagrangiana y luego la formulación
    Hamiltoniana:
    \begin{enumerate} 
      \item si la cantidad de movimiento de la partícula es $p_0$ a tiempo
        $t_0$, cuando las paredes están separadas por una distancia $x_0$ entre
        sí, encuentre la cantidad de movimiento de la partícula a tiempo
        $t>t_0$, suponiendo que las colisiones con las paredes son
        perfectamente elásticas y el movimiento es no-relativista;
      \item cuando las paredes están separadas por una distancia $x$, cuál debe
        ser la fuerza promedio que debe ser aplicada a cada pared para que se
        muevan a velocidad constante hacia el centro.
    \end{enumerate}

  \item {\bf{Fuerzas centrales}}

    Escriba el problema del movimiento bajo fuerzas centrales de dos masas
    puntuales en la formulación Hamiltoniana, eliminando las variables
    cíclicas.

  \item {\bf{Lagrangiano con término de interacción}}

    El Lagrangiano de un sistema puede ser escrito como: 
    \[ \mathcal{L} = a \dot x^2 + b \frac{\dot y}{x} + c \dot x \dot y + f y^2
    \dot x \dot z + g \dot y - k \sqrt{x^2 + y^2}, \]
    donde $a$, $b$, $c$, $f$, $g$ y $k$ son constantes. Obtenga el Hamiltoniano
    del sistema, las ecuaciones de movimiento y diga que cantidades son
    conservadas. 

  \item {\bf{Lagrangiano numérico, I}}

    El Lagrangiano de un sistema puede ser escrito como: 
    \[ \mathcal{L} = \dot q_1^2 + \frac{\dot q_2^2}{a + b q_1^2}  + k_1 q_1^2 +
    k_2 \dot q_1 \dot q_2, \]
    siendo $a$, $b$, $k_1$ y $k_2$ constantes. 
    \begin{enumerate}
      \item Encuentre las ecuaciones de movimiento del sistema en la
        formulación Hamiltoniana
      \item Diga si el sistema tiene alguna magnitud conservada. 
      \item Suponiendo que todas las constantes son iguales a $1$, usando el
        algoritmo RK4 resuelva numéricamente las ecuaciones de movimiento
        obtenidas, con condiciones iniciales ($t_0=0$), $q_1(t_0)=q_2(t_0)=0$,
        $\dot q_1(t_0)=\dot q_2(t_0)=1$ (reescriba las condiciones iniciales
        para $p_i$).
      \item Haga un gráfico de $q_1$, $q_2$, $p_1$ y $p_2$ como función del
        tiempo.
      \item Haga un gráfico de la trayectoria del sistema en el espacio de
        fases. 
    \end{enumerate}

  \item {\bf{Péndulo simple no tan simple}}

    Imagine que el punto de sujeción de un péndulo simple, de masa $m$ y
    longitud $l$, se dispone de forma tal que el mismo puede moverse a lo largo
    de una parábola $y=a x^2$ en el plano vertical ($a>0$). Encuentre el
    Hamiltoniano del sistema y las ecuaciones de movimiento del péndulo y del
    punto de sujeción. Luego, suponga $a=0$ y eligiendo una condición inicial
    no trivial ($z_0>0$, $\theta_0>0$), resuélvalas numéricamente. 
  
  \item {\bf{Dos resortes}} 
    
    Una partícula de masa $m$ puede moverse en 1 dimensión bajo la acción de
    dos resortes de constantes elásticas $k_1$ y $k_2$ respectivamente. Cada
    resorte está sujeto en un extremo a una pared fija y en el otro a la
    partícula. La distancia de separación de las paredes es $a$. Los resortes
    obedecen la ley de Hooke y tienen longitud de relajación igual a 0 (es
    decir, siempre están tensionados).

    \begin{enumerate}
      \item Usando como coordenada generalizada $q$ la posición $x$ de la
        partícula medida desde uno de los puntos fijos (de forma que estos se
        encuentran en $x=0$ y $x=a$), encuentre el Lagrangiano y derive el
        correspondiente Hamiltoniano. Obtenga las ecuaciones de movimiento y
        diga si la energía es conservada y si el Hamiltoniano lo es.
      \item Introduzca una nueva coordenada generalizada: 
        \[ Q = q - b \sin \omega t, \quad \quad b=\frac{k_2}{k_1 + k_2} a,\]
        y responda: ¿Cuál es el nuevo Lagrangiano como función de $Q$? ¿Cuál es
        el Hamiltoniano correspondiente? ¿Es la energía una magnitud
        conservada? ¿y el Hamiltoniano?
    \end{enumerate}

  \item {\bf{Lagrangiano numérico, II}} 
    
    Considere un Lagrangiano de la forma: 
    \[ 
      \mathcal{L} = \frac12 m \left ( \dot x^2 - \omega^2 x^2  \right )
       \exp( \gamma t), 
    \] 
    donde la partícula de masa $m$ posee un movimiento unidimensional.
    Suponiendo que todas las constantes son positivas:
    
    \begin{enumerate}
      \item encuentre las ecuaciones de movimiento en la formulación
        Lagrangiana; 
      \item interprete las ecuaciones mediante la interpretación física de las
        fuerzas actuando sobre la partícula;
      \item Encuentre los momentos canónicos y construya el Hamiltoniano. ¿Es
        el Hamiltoniano una constante de movimiento? 
      \item Para las condiciones iniciales $x(0)=0$ y $\dot x=0$, y suponiendo
        un sistema de unidades donde $\omega=\gamma=1$, resuelva numéricamente
        las ecuaciones de movimiento y verifique a que valor tiene $x(t)$
        para valores grandes de $t$.
    \end{enumerate}

  \item {\bf{Condición simpléctica}}

    Mostrar que la transformación 
    \begin{eqnarray*}
      Q &=& q \cos \alpha - p \sin \alpha,\\
      P &=& q \sin \alpha + p \cos \alpha,
    \end{eqnarray*}
    verifica la condición simpléctica para cualquier valor del parámetro
    $\alpha$. Luego, encuentre una función generatriz $F$ para la
    transformación. Explique el significado físico de esta transformación
    cuando $\alpha=0$ y cuando $\alpha=\pi/2$.

  \item {\bf{Transformación de punto}}

    Las ecuaciones de transformación entre dos sistemas de coordenadas son: 
    \begin{eqnarray*}
      Q &=& \log \left ( 1 + \sqrt{q} \cos p \right ),\\
      P &=& 2 \left ( 1 + \sqrt{q} \cos p \right ) \sqrt{q} \sin p.
    \end{eqnarray*}
    Muestre que si $p$ y $q$ son variables canónicas, entonces $Q$ y $P$
    también lo son. Luego verifica que esta transformación es generada por:
    \[ F_3 = - \left ( e^Q -1 \right )^2 \tan p.\]

  \item {\bf{Condiciones directas}}

    Muestre por evaluación directa que las condiciones directas para una
    transformación canónica son equivalentes a la condición simpléctica
    expresada en la forma: 
    \[ \mathbb{J M} = \mathbb{M^{-1} J}.\]

  \item {\bf{Evaluación simpléctica}}

    Utilizando la condición simpléctica, muestre que la transformación de
    coordenadas: 
    \begin{eqnarray*}
      Q_1 &=& q_1^2, \\
      P_1 &=& \frac{p_1 \cos p_2 - 2 q_2}{2 q_1 \cos p_2}, \\
      Q_2 &=& q_2 \sec p_2, \\
      P_2 &=& \sin p_2 - 2 q_1,
    \end{eqnarray*}
    es canónica. Luego, encuentre una posible función generatriz para esta
    transformación.

  \item {\bf{Evaluación simpléctica}}

    Para la transformación de punto en un sistema con dos grados de libertad:
    \begin{eqnarray*}
      Q_1 &=& q_1^2,\\
      Q_2 &=& q_1 + q_2,
    \end{eqnarray*}
    encuentre las ecuaciones de transformación más generales posibles para
    $P_1$ y $P_2$ de manera que la transformación general sea canónica. Muestre
    que para una dada elección de las funciones $P_1$ y $P_2$, el Hamiltoniano
    \[ 
      \mathcal{H} = \left ( \frac{p_1 - p_2}{2 q_1} \right )^2 + p_2 + (q_1 +
      q_2)^2
    \]
    puede ser transformado en un nuevo Hamiltoniano donde tanto $Q_1$ como
    $Q_2$ sean cíclicos. Luego, obtenga $q_1$, $q_2$, $p_1$ y $p_2$ como
    función del tiempo para condiciones iniciales genéricas.

  \item {\bf{Oscilador armónico}}

    Muestre que la transformación 
    \begin{eqnarray*}
      Q &=& p + i a q,\\
      P &=& \frac{p - i a q}{2 i a},
    \end{eqnarray*}
    es una transformación canónica y encuentre una función generatriz. Luego,
    use la transformación para resolver el oscilador armónico.

  \item {\bf{Corchetes de Poisson}}

    Muestre que si el Hamiltoniano y una función $F$ son constantes de
    movimiento, entonces la enésima derivada parcial de $F$ con respecto al
    tiempo, $\partial^n F/\partial t^n$, también lo es. Luego, considere el
    movimiento de una partícula libre de masa $m$. Sabemos que en este caso el
    Hamiltoniano del sistema es una magnitud conservada, y existe una constante
    de movimiento
    \[ F = x - \frac{p t}{m}.\]
    Muestre por cálculo directo que la derivada parcial de $F$ respecto al
    tiempo es una constante de movimiento y que cumple con la condición
    $[\mathcal{H}, F]$.

\end{enumerate}
\end{document}
